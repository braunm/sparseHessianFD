\documentclass{article}
\usepackage[T1]{fontenc}
\usepackage[utf8]{inputenc}
\usepackage{amsmath}
\usepackage{amsfonts}
\usepackage{subcaption}
\usepackage{algorithm}
\usepackage{algorithmic}
\usepackage{tabularx}
\usepackage{dcolumn}
\usepackage{booktabs}
\usepackage{etoolbox}
\usepackage{placeins}

\newcolumntype{d}[1]{D{.}{.}{#1}}
\newcommand{\pkg}[1]{\emph{#1}}
\newcommand{\proglang}[1]{\textsf{#1}}
\newcommand{\filename}[1]{\textit{#1}}
\newcommand{\code}[1]{\texttt{#1}}
\newcommand{\func}[1]{\code{#1}}
\newcommand{\class}[1]{\textsl{#1}}
\newcommand{\funcarg}[1]{\code{#1}}
\newcommand{\variable}[1]{\code{#1}}
\newcommand{\method}[1]{\func{#1}}


 \newcommand{\df}[3]{\mathsf{d}^{#1}f(#2;#3)}
 \newcommand{\parD}[3]{\mathsf{D}^{#1}_{#2}#3}
\newcommand{\hess}[2]{\mathsf{H}_{#1}#2}
\newcommand{\hessLT}[2]{\mathsf{L}_{#1}#2}
\newcommand{\Mat}[1]{\mathbf{#1}}
\newcommand{\Real}[1]{\mathbb{R}^{#1}}

\newenvironment{revQuote}{\itshape}{\vspace{\baselineskip}}
\newenvironment{response}{\normalfont}{\vspace{\baselineskip}}


\usepackage{geometry}
\geometry{margin=1.5in}

\begin{document}

{\bfseries
\begin{center}
  Manuscript JSS2717\\
 ``sparseHessianFD: An R Package for Estimating Sparse
 Hessian Matrices''\\
 \vspace{\baselineskip}
 Response to reviewers
\end{center}
}
\vspace{\baselineskip}

I would like to express my sincere appreciation for the time and effort that
the reviewer invested in the review.  It is obvious to me that the
reviewer went through the paper and code in great detail.  In
particular, I want to thank the reviewer for bringing the complex step
method to my attention.  I \emph{never} use complex numbers or
functions in my own work, so I typically ignore anything with the word
"complex" in it.  But this review prompted me to learn something new.
That is always fun.

I am happy to write that I was able to incorporate a large majority of
the reviewer's suggestions in the revised paper and package. For the
others, I hope that the editor and reviewer will accept my
explanations and excuses.  Of course, if there are any strong
objections to how I responded to the reviewer's concerns, I am happy
to make further changes in advance of publication.

\vspace{\baselineskip}

\begin{enumerate}
 \setcounter{enumi}{-1}

\item \begin{revQuote}
I had a minor problem checking the package tarball:

{\normalfont (error messages related to algorithm.sty removed).}

I don't remember if there is an easy way to distribute ``special'' sty files,
or if you must force the user to install them. If the sty is not really
critical it might be better to omit it.  
  
\end{revQuote}

\begin{response}
I suspect that the error is a result of the \pkg{algorithm} package
not being present in the reviewer's LaTeX installation.  I would like
to continue to use this package,
because it does allow for nicely formatted algorithms.  I believe that this is
a relatively common package, and one that should be easily installed
through CTAN.  However, if there is a strong
objection, I may be able to find another way to format the algorithms.
\end{response}

\item \begin{revQuote}
I am not familiar with reference classes, but for most users I think it
should not be important that the package uses them. However, as a user, I
find it disconcerting that sparseHessianFD() is used as a (constructor)
function (eg. p12, and in the example in ?sparseHessianFD) but is not
documented in the help with 'usage' and 'arguments' as is usual for
functions. There seems to be little guidance in the usual R places about how
reference classes should be documented. Perhaps you could seek some guidance
from the community on this point.
\end{revQuote}


\begin{response}

  Note to me:  issue here is that there is no documentation for
  \func{sparseHessianFD()} as a function.  Good idea to create a
  documentation page for that.

  
\end{response}



\item \begin{revQuote}
It is pointed out (p2) that requirements are burdensome and emphasized
that the package is not appropriate for all uses. Also, the conditions 1-5
are fairly stringent. The reader is left with the impression that there may
be few applications. I think the value of the package could be promoted
better. Perhaps a small list of different types or applications of
hierarchical problems could be given, or an indication of other classes of
problems where the conditions would apply.
    
  \end{revQuote}

  \begin{response}

I included this section primarily as a way to avoid overselling the
method, but the point that the reader might be scared away by this list of
restrictions is well-taken.  In the revision, I removed the list of
conditions, and wove them into the text.  The result takes a more
positive tone that in the original submission.

\end{response}

\item \begin{revQuote}
Condition 5 might be relaxed to a local condition in a neighborhood of
the evaluation point, rather than a global condition, but I do not know if
that would have any practical value.
  \end{revQuote}

\begin{response}

  The list of conditions is no longer in the paper, but to this point,
  I added a defintion of a ``structural zero'' in the third
  paragraph. The sparsity pattern represents the structure of the
  Hessian, not its value, so any element that \emph{could} be zero,
  but does not have to be zero, is not a structural zero.  Thus, the
  condition is global and not local.

  Note that other than
  a potential loss of efficiency, there is no disadvantage to erring
  on the side of an element being non-zero.  
\end{response}

\item \begin{revQuote}
Section 2.1 p7 I think could be made more easily readable by adding a few
more hints about dimension, and more care about use of the term 
"coefficients vector", for example:

  - 'continuous covariates $x_i$,' -> 'continuous covariates $x_i \in
\Real{k}$,'

  - 'heterogeneous coefficient vector $\beta_i$' -> 'heterogeneous
coefficient vector $\beta_i \in \Real{k}$'

  - 'The coefficients are distributed' -> 'The coefficient vectors
$\beta_i$ are distributed'

  \end{revQuote}

\begin{response}
  Thank you for the suggestion.  Changes have been made throughout the paper.
\end{response}


\item \begin{revQuote}
p7 following "the cross-partial derivatives"
$\hess{\beta_i,\beta_j}{}=\parD{2}{\beta_i, \beta_j}{}=0$ for all $i\neq
j$.  

  Is this a definition of $\hess{\beta_i,\beta_j}{}$ in terms of  $\parD{2}{}{}$
or a statement of equality?
  
  Also, I am confused by the single subscript $\beta_i$ to \\hess here, but a
double subscript just above in "Thus, $\hess{\beta_{ik},\mu_k}{}\neq 0$".
  \end{revQuote}

\begin{response}
  
\end{response}


\item \begin{revQuote}
 I think possibly (16) is "banded" and (17) "block arrow" rather than the
reverse which is indicated.
  \end{revQuote}

\begin{response}
  
\end{response}


\item \begin{revQuote}
Code in the paper prior to table 4 needs to be copied from the
vinettes/sparseHessianFD.Rnw file. For those trying to reproduce results it
would be nice if this where mentioned in the file replication.R.
  \end{revQuote}

\begin{response}
  
\end{response}


\item \begin{revQuote}
p9, last line. The R> at the beginning of
 
   R> obj <- sparseHessianFD(x, fn, gr, rows, cols, ...)

suggests that this is code that can be entire at the command line, but ...
causes an error when entered (and the arguments have not been defined at
this point in the vignette. Instead it should be indicated as the usage
syntax.
  \end{revQuote}

\begin{response}
  
\end{response}


\item \begin{revQuote}
p10. "where ... represents all other named arguments"
   I think the usual usage is the ... represents the arguments other than
the "named" ones, so it is probably better to just say "other arguments".
  \end{revQuote}

\begin{response}
  
\end{response}


\item \begin{revQuote}
On p12, the function calls all.equal(f, true.f) and all.equal(gr,
true.grad) are comparing f and gr calculated with the exact calculation, so
the difference is zero:

\begin{verbatim}
print(obj$fn(P) - binary.f(P, 
      data=binary, priors=priors, order.row=order.row), digits=20)
[1] 0

print(max(abs(obj$gr(P) - binary.grad(P, 
         data=binary, priors=priors,
order.row=order.row))),digits=20)
[1] 0

\end{verbatim}

It is not clear to me whether obj\$fn() and obj\$gr() use code as in the true
functions or a modified version using sparse techniques. Some further
clarification would be helpful.

On the other hand, all.equal(hs, true.hess) is comparing a true analytic
calculation with a first order simple difference aproximation using the true
gradient function:
\begin{verbatim}
max(abs(  obj$hessian(P)
     - binary.hess(P, data=binary, priors=priors, order.row=order.row)))
[1] 2.786891e-06
\end{verbatim}

which might also be mentioned in the text. (Really just for exposition
purposes, after all, it is almost the main purpose of the package.)
p10. "where ... represents all other named arguments"
   I think the usual usage is the ... represents the arguments other than
the "named" ones, so it is probably better to just say "other arguments".
  \end{revQuote}

\begin{response}
  
\end{response}


\item \begin{revQuote}
I think it would be instructive to add some of the following comparisons
with the above on p12.  The package numDeriv function hessian by default
does a second order Richardson approximation using the true function value
approximation. This involves a very large number of function evaluations in
an attempt to obtain some accuracy, but the accuracy is limited by being an
approximation of a second difference:
\begin{verbatim}
max(abs(
     numDeriv::hessian( binary.f, P, method="Richardson", 
                    data=binary, priors=priors,
order.row=order.row) 
   - binary.hess(P, data=binary, priors=priors, order.row=order.row)))
[1] 0.0001610595

\end{verbatim}

Since the hessian is the first difference of the gradient, which is the
calculation used by obj\$hessian() in sparseHessianFD, one could also use the
function numDeriv::jacobian:

\begin{verbatim}
max(abs( 
    numDeriv::jacobian( binary.grad, P, method="Richardson", 
                   data=binary, priors=priors,
order.row=order.row)
  - binary.hess(P, data=binary, priors=priors, order.row=order.row)))
[1] 3.268224e-07
\end{verbatim}

This is still doing the calculation intensive Richardson approximation. The
calculation which would seem to most closely resemble what is done by
obj\$hessian() is

\begin{verbatim}
max(abs( 
    numDeriv::jacobian( binary.grad, P, method="simple", 
                   data=binary, priors=priors,
order.row=order.row)
  - binary.hess(P, data=binary, priors=priors, order.row=order.row)))

[1] 0.0008255852

\end{verbatim}


Another very interesting comparison is

\begin{verbatim}
max(abs( 
  numDeriv::jacobian( binary.grad, P, method="complex", 
                  data=binary, priors=priors,
order.row=order.row)
 - binary.hess(P, data=binary, priors=priors, order.row=order.row)))

[1] 7.105427e-15
\end{verbatim}

The complex step derivative provides extremely accurate approximations with
a  number of function evaluation similar to the simple method. (This does
not seem to be anticipated by footnote 1 in the paper.) However, the method
imposes some serious requirements on the function. (Something like complex
analytic even though the user may only be interested in the real part.) The
code also has to accept complex arguments and return the complex result.
Fortunately most R primitive work with complex numbers so the code
requirement may happen accidentally, which can be partly verified by

binary.grad(P + 0+1i, data=binary, priors=priors, order.row=order.row)

returning a complex result. (This does not rule out all possible problems.)

As I recall, sums, multiplication, and exponentiation are all complex
analytic, so it would not be too surprising if the example in the paper is
too, but I have not analyzed that. However, based on the result being very
good, it seems highly likely.
  \end{revQuote}

\begin{response}
  
\end{response}


\item \begin{revQuote}
A possible extension to the package would be to implement the complex
method in the sparse code. The function numDeriv:::jacobian.default
implements both simple and complex, so provides a good comparison of the
necessary (non-sparse) computation. 


    
  \end{revQuote}

\begin{response}
  
\end{response}


\item \begin{revQuote}
 p.13  l. -7  Figure 3b  should be Figure 3c
  \end{revQuote}

\begin{response}
  
\end{response}


\item \begin{revQuote}
 The file replication.R does not set the RNG seed. This may not be too
important if only times are generated, but will be if resulting values are
included.
    
  \end{revQuote}

\begin{response}
  
\end{response}


\item \begin{revQuote}
Table 4 and 5. Some OS, processor, and memory details are helpful to put
timing results in context.
  \end{revQuote}

\begin{response}
  
\end{response}


\item\begin{revQuote}
Table 4 is really not the proper comparison. I think a comparison with
       numDeriv::jacobian( binary.grad, P, method="simple", 
                   data=binary, priors=priors,
order.row=order.row)

really serves to highlight the improvement of the sparse calculation because
it is a valid comparison. Even though the results are not as exaggerated,
they are still important:

(My laptop is a  Intel(R) Core(TM) i5-3337U CPU @ 1.80GHz,  4GB RAM, SSD
swap, running Mint variant of Ubuntu 14.04.2 LTS.)

\begin{verbatim}
run.par <- FALSE

(Note this is using obj$gr() rather than binary.grad(). I am not sure if
that makes a difference.)

run_test_tab4b <- function(Nk, reps=50) {
    ## Replication function like Table 4  with jacobian of grad and
"simple"
    N <- as.numeric(Nk["N"])
    k <- as.numeric(Nk["k"])
    data <- binary_sim(N, k, T=20)
    priors <- priors_sim(k)
    F <- make_funcs(D=data, priors=priors)
    nvars <- N*k+k
    M <-
as(Matrix::kronecker(Matrix::Diagonal(N),Matrix(1,k,k)),"nMatrix") %>%
      rBind(Matrix(TRUE,k,N*k)) %>%
      cBind(Matrix(TRUE, k*(N+1), k)) %>%
      as("nMatrix")
    pat <- Matrix.to.Coord(tril(M))
    X <- rnorm(nvars)
    obj <- sparseHessianFD(X, F$fn, F$gr, pat$rows, pat$cols)

    bench <- microbenchmark(
        numDeriv = numDeriv::jacobian(obj$gr, X, method="simple"), 
        df = obj$gr(X),
        sparse = obj$hessian(X))
    vals <- plyr::ddply(data.frame(bench), "expr",
                  function(x)
return(data.frame(expr=x$expr,
                                               
time=x$time,
                                               
rep=1:length(x$expr))))
    res <- data.frame(N=N, k=k,
                      bench=vals)
    cat("Completed N = ",N,"\tk = " , k ,"\n")
    return(res)
}

cases_tab4b <- expand.grid(k=c(2,3,4),
                     N=c(9, 12, 15))
runs_tab4b <- plyr::adply(cases_tab4b, 1, run_test_tab4b, reps=20,
.parallel=run.par)

tab4b <-  mutate(runs_tab4b, ms=bench.time/1000000) %>%
  select(-bench.time) %>%
  spread(bench.expr, ms) %>%
  gather(method, hessian, c(numDeriv, sparse)) %>%
  mutate(M=N*k+k, hessian.df=hessian/df) %>%
  gather(stat, time, c(hessian, hessian.df)) %>%
  group_by(N, k, method, M, stat)  %>%
  summarize(mean=mean(time), sd=sd(time)) %>%
  gather(stat2, value, mean:sd) %>%
  dcast(N+k+M~stat+method+stat2,value.var="value") %>%
  arrange(M)

tab4b 
   N k  M hessian_numDeriv_mean hessian_numDeriv_sd hessian_sparse_mean
1  9 2 20              5.731754          
0.6759421            1.713736
2 12 2 26              7.577777          
0.9481714            1.756165
3  9 3 30              8.763990          
1.0358481            2.348153
4 15 2 32              9.553452          
1.1875816            1.844192
5 12 3 39             11.583360          
1.3371329            2.402663
6  9 4 40             11.584188          
1.2045539            2.978028
7 15 3 48             14.638763          
1.5469615            2.547145
8 12 4 52             15.612216          
2.5401431            3.086010
9 15 4 64             19.672496          
2.7303771            3.055931
  hessian_sparse_sd hessian.df_numDeriv_mean hessian.df_numDeriv_sd
1        0.07532715                
19.02359               3.353629
2        0.07522725                
24.31841               4.715941
3        0.15790507                
28.47628               4.366788
4        0.30441636                
30.52029               4.533416
5        0.08570303                
35.98930               6.033750
6        0.55389180                
36.30998               6.007112
7        0.61666543                
44.46145               6.206399
8        0.69277761                
48.73135               9.180011
9        0.12617835                
60.27624               9.530503
  hessian.df_sparse_mean hessian.df_sparse_sd
1               5.687322            0.7812959
2               5.619639            0.7445722
3               7.615163            0.7901088
4               5.884283            1.0339672
5               7.455115            0.8976474
6               9.329878            2.0531265
7               7.734390            2.0592415
8               9.637077            2.4535258
9               9.356261            0.6168655

\end{verbatim}


  \end{revQuote}
  
  \begin{response}
    
  \end{response}

\item\begin{revQuote}
It is possible to do a larger example with this comparison:

(On my laptop the next took about 30 hrs of which 24 was for the last,
N=2500, k=8 comparison.)
\begin{verbatim}
cases_tab4b5 <- expand.grid(k=c(2,4,8),
                     N=c(10, 100, 1000, 2500))
runs_tab4b5 <- plyr::adply(cases_tab4b5, 1, run_test_tab4b, reps=20,
.parallel=run.par)

tab4b5 <-  mutate(runs_tab4b5, ms=bench.time/1000000) %>%
  select(-bench.time) %>%
  spread(bench.expr, ms) %>%
  gather(method, hessian, c(numDeriv, sparse)) %>%
  mutate(M=N*k+k, hessian.df=hessian/df) %>%
  gather(stat, time, c(hessian, hessian.df)) %>%
  group_by(N, k, method, M, stat)  %>%
  summarize(mean=mean(time), sd=sd(time)) %>%
  gather(stat2, value, mean:sd) %>%
  dcast(N+k+M~stat+method+stat2,value.var="value") %>%
  arrange(M)

tab4b5
tab4b5
      N k     M hessian_numDeriv_mean hessian_numDeriv_sd
hessian_sparse_mean
1    10 2    22          6.363689e+00       
6.585771e-01            1.795344
2    10 4    44          1.273715e+01       
1.254662e+00            2.934588
3    10 8    88          2.608751e+01       
1.803375e+00            5.659668
4   100 2   202          1.021658e+02       
3.439251e+00            2.938642
5   100 4   404          2.162431e+02       
1.289940e+01            5.622877
6   100 8   808          4.846450e+02       
1.946854e+01           12.922397
7  1000 2  2002          5.511146e+03       
1.918534e+02           14.694924
8  1000 4  4004          1.194815e+04       
1.499518e+02           29.332463
9  2500 2  5002          3.132878e+04       
4.934896e+02           33.393927
10 1000 8  8008          2.835954e+04       
6.959400e+02           76.881070
11 2500 4 10004          6.874300e+04       
2.197321e+02           68.318820
12 2500 8 20008          8.615465e+05       
9.316236e+04         1806.766254
   hessian_sparse_sd hessian.df_numDeriv_mean hessian.df_numDeriv_sd
1       6.056685e-01                
21.19760               2.433792
2       5.508396e-01                
41.16514               5.648203
3       1.265859e+00                
79.97503               8.748075
4       7.409767e-02               
193.27447              11.790089
5       1.210575e+00               
388.74262              31.184602
6       1.168307e+01               
758.39623             121.225037
7       9.293258e-01              
2042.30264             168.538055
8       1.343366e+00              
4099.05785             221.910206
9       1.377541e+00              
5081.52982             473.962083
10      2.408524e+01              
8323.91140             516.709041
11      2.646135e+00             
10315.16422             832.305746
12      2.272944e+03             
77200.22397           42790.360882
   hessian.df_sparse_mean hessian.df_sparse_sd
1                5.984750            2.1093324
2                9.479244            1.9844323
3               17.337516            4.1347369
4                5.556561            0.2851630
5               10.115250            2.3189536
6               20.160163           17.8116929
7                5.448765            0.5700265
8               10.061368            0.6629470
9                5.416287            0.5421972
10              22.563990            7.1995778
11              10.249990            0.8971753
12             151.538778          217.2031623


I think the complex step takes a similar amount of time, but produces a more
accurate result.

\end{verbatim}

  \end{revQuote}
  
  \begin{response}
    
  \end{response}

\item\begin{revQuote}

 p.16  l. -7 "to to compute" -> "to compute"


  \end{revQuote}
  
  \begin{response}
    
  \end{response}

\item\begin{revQuote}

While that package is useful, and reasonably demonstrated in the paper,
I think it would be nice to expand the paper in some ways that might be
deemed more "original research". Some possibilities are:

 - Explain and try to assess how much of the speedup is due to simple
sparseness and how much is due to the "sparse patern" (p2) allowing for
perturbing multiple elements together. (I think these are related but
slightly different?)
 
 - Try to assess how much of the speedup is due to reduce computing demand
and how much is due to different memory demand. (I had the impression in the
larger problems with numDeriv that my computer started to use swap space,
which resulted in a big slow down.)
 
 - Consider implementing a complex step method, and do a comparison.
 
 - Assess the difference when multiple CPUs are used. (run.par == FALSE vs
run.par == TRUE)

  \end{revQuote}
  
  \begin{response}
    
  \end{response}



  


\end{enumerate}

\end{document}