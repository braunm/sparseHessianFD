\subsubsection[Storage formats]{Storage formats}

Matrices with a ``triplet'' format store each non-zero element as a
(row, column, value) set.  The values are either numeric or logical,
for \class{d*TMatrix} or \class{l*TMatrix} matrices, respectively.
Pattern matrices of class \class{n*TMatrix} store only row and column
indices, but no values at all.

Row-compressed (CSR) and column-compressed (CSC) formats are more
efficient alternatives.  Instead of storing the row and column indices, CSC
matrices store only the row indices, and ``pointers''
to the values in the row index and value vectors that start
each column\footnote{For the CSR format, switch references to columns and rows
in the preceding sentence.}.

For example, let's construct a numeric matrix $\Mat{A}$ in triplet form using
the \func{sparseMatrix} function. The arguments \funcarg{i},
\funcarg{j} and \funcarg{x} are the row and column indices, and
values, respectively, of the non-zero elements.  By default, \func{sparseMatrix} expects the input vectors \funcarg{i} and \funcarg{j} to
have one-based indexing, and stores $\Mat{A}$ in CSC format.  We can then
look at the values directly, and the pointers (using the familiar
one-based indexing scheme) using \func{Matrix.to.Pointers}.

<<trip>>=
library("sparseHessianFD")
A <- sparseMatrix(i=c(1,4,2,3,3,4), j=c(1,1,2,2,3,4), x=c(6,7,8,9,10,11))
printSpMatrix(A)
Matrix.to.Pointers(A, values=TRUE, order="column", index1=TRUE)
@


The slot \funcarg{x} contains the matrix values exactly as we entered
them.  The column indices are not stored explicitly, but the pointers in \variable{jpntr}
determine where in the sequence of values and row indices (\variable{iRow}) the next
column will start. The first pointer is always 1, so the first element in column 1 is in
 row \code{iRow[1]=1}, and has value \code{x[1]=6}.  Since \code{jpntr[2]=3}, \code{x[2]=7} is still in
the first column, in row \code{iRow[2]=4}, so $\Mat{A}_{41}=7$.  Elements 3 and 4 in
\code{x} and \code{iRow} refer to values in column 2, since
\code{jpntr[3]=5} means that column 3 does not begin until the element
5. Hence, $\Mat{A}_{22}=x_3=8$ and $\Mat{A}_{32}=x_4=9$.
Column 3 begins with $\Mat{A}_{33}=x_5=10$ (\code{iRow[5]=3} gives us the
row index).  Continuing,
$\Mat{A}_{44}=x_6=11$.  The final value in
\variable{jpntr} is always the number of non-zeros in the matrix.  The differences between
adjacent elements in \variable{jpntr} are the number of non-zeros in each column of $\Mat{A}$.

There are two important advantages to storing sparse matrices in a
compressed format.  First, the storage requirements are usually smaller.  A
matrix with $M$ columns and $N$ non-zeros requires $3N$ elements in a
triplet format, but only $2N+M+1$ elements in a compressed format.
Second, computation on the matrix is more efficient, since all elements in a
single column (CSC) or row (CSR) are stored in contiguous memory, in
order.  In contrast, matrix triplets could be entered or stored in any
order.
